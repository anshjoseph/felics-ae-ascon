\documentclass{article}

\usepackage[utf8]{inputenc}
\usepackage{authblk}
\usepackage[T1]{fontenc}
\usepackage{hyperref}


\title{
  FELICS-AE: a framework to benchmark lightweight authenticated block ciphers
}

\author[*]{Kévin Le Gouguec}

\affil[*]{
  Airbus CyberSecurity -
  ZA Clef Saint-Pierre,
  1 Bd Jean Moulin,
  CS 40001,
  MetaPole,
  78996 ÉLANCOURT Cedex -
  France -
  \href{mailto:kevin.legouguec@airbus.com}{kevin.legouguec@airbus.com}
}


\begin{document}

\maketitle

\section{Introduction}
\label{sec:intro}

The CAESAR competition~\cite{CAESAR:submissions} and the NIST
Lightweight Cryptography Standardization Process~\cite{NIST:LWC} have
brought to light several new Authenticated Encryption with Associated
Data (AEAD) schemes dedicated to ``lightweight'' use-cases.  In these
use-cases, target devices are strongly constrained in terms of
computing resources: they have limited volatile memory (RAM) and
non-volatile memory (ROM), their processors operate at low frequencies
and feature few registers, they may only be able to draw power from a
battery that can neither be recharged nor replaced, etc.

These devices thus have very few resources to spare on security.  This
implies that the aforementioned algorithms must be selected not only
for their robustness, but also according to their efficiency.  Given
two encryption schemes with equivalent security, the scheme which
leaves the target device more resources to perform its designed
function should be preferred.

Thus measuring the performance of these algorithms is an integral part
of the selection process carried out in~\cite{NIST:LWC}.  In this
paper, we present FELICS-AE, an adaptation of the FELICS
framework~\cite{FELICS:paper} dedicated to authenticated encryption
schemes, which we use to assess the performance of our candidate
\textsc{Lilliput-AE}~\cite{NIST:Lilliput-AE}.

First, in section~\ref{sec:background}, we will present the original
FELICS framework.  We will then present FELICS-AE in
section~\ref{sec:felics-ae}, going over our work to adapt the
framework and explaining how to set it up, measure algorithm
performance, and add new encryption schemes.  We will present the
results we obtained so far in section~\ref{sec:results}.  To conclude,
we will mention possible improvements for FELICS-AE in
section~\ref{sec:future}.

\section{Background: the FELICS framework}
\label{sec:felics}

The FELICS framework\cite{FELICS:paper} includes a collection of
implementations of encryption algorithms in C, as well as a set of
shell scripts which measure the performance of these algorithms on
various microcontrollers representative of ``Internet of Things''
(IoT) situations.

\subsection{Supported devices}
\label{sec:felics/devices}

FELICS supports the following microcontrollers (\textbf{bolded} words
denote the codenames used within the framework):

\begin{itemize}
\item 8-bit \textbf{AVR} ATmega128,
\item 16-bit \textbf{MSP}430F1611,
\item 32-bit \textbf{ARM} Cortex-M3.
\end{itemize}

The AVR and MSP430 platforms are entirely simulated, which allows one
to measure algorithm performance on these microcontrollers without
physically owning them.  To measure performance on ARM, however,
FELICS requires an Arduino Due board, as well as a J-Link probe.

FELICS can also benchmark algorithms for x86 architectures (codenamed
\textbf{PC}).

\subsection{Metrics}
\label{sec:felics/metrics}

\paragraph{Code size}

FELICS adds up the \texttt{text} and \texttt{data} sections of an
implementation's compiled object code, as reported by the GNU
\texttt{size} program, to measure how much space the algorithm's
footprint on non-volatile memory.

\paragraph{RAM}

To measure the working memory needed by an algorithm, the framework
runs the implementation through a debugger, spraying a known pattern
on the stack before execution and counting how many bytes were
modified after execution.  This figure is added to the object code's
\texttt{data} section.

\paragraph{Execution time}

For simulated devices, FELICS relies on the simulator to keep track of
the number of clock cycles spent on encryption.  For other devices,
FELICS uses specialized assembly instructions to get this information.

\subsection{Distribution}
\label{sec:felics/dist}

\subsection{Algorithm instrumentation}
\label{sec:felics/adding-algos}

\section{FELICS for Authenticated Encryption}
\label{sec:felics-ae}

FELICS was initially developed to measure the performance of block
ciphers and stream ciphers.  In this section, we describe the changes
we made to adapt the framework to AEAD algorithms; we then describe
how it can be used and extended.

\subsection{Changes from FELICS}
\label{sec:felics-ae/diff-felics}

\subsection{Usage}
\label{sec:felics-ae/usage}

\section{Results}
\label{sec:results}

\section{Future work}
\label{sec:future}

\bibliographystyle{plain}
\bibliography{references}

\end{document}
