\documentclass{article}

\usepackage[utf8]{inputenc}
\usepackage{authblk}
\usepackage[T1]{fontenc}
\usepackage{hyperref}


\title{
  FELICS-AE: a framework to benchmark lightweight authenticated block ciphers
}

\author[*]{Kévin Le Gouguec}

\affil[*]{
  Airbus CyberSecurity -
  ZA Clef Saint-Pierre,
  1 Bd Jean Moulin,
  CS 40001,
  MetaPole,
  78996 ÉLANCOURT Cedex -
  France -
  \href{mailto:kevin.legouguec@airbus.com}{kevin.legouguec@airbus.com}
}


\begin{document}

\maketitle

\section{Introduction}
\label{sec:intro}

The CAESAR competition~\cite{CAESAR:submissions} and the NIST
Lightweight Cryptography Standardization Process~\cite{NIST:LWC} have
brought to light several new Authenticated Encryption with Associated
Data (AEAD) schemes dedicated to ``lightweight'' use-cases.  In these
use-cases, target devices are strongly constrained in terms of
computing resources: they have limited volatile memory (RAM) and
non-volatile memory (ROM), their processors operate at low frequencies
and feature few registers, they may only be able to draw power from a
battery that can neither be recharged nor replaced, etc.

These devices thus have very few resources to spare on security.  This
implies that the aforementioned algorithms must be selected not only
for their robustness, but also according to their efficiency.  Given
two encryption schemes with equivalent security, the scheme which
leaves the target device more resources to perform its designed
function should be preferred.

Thus measuring the performance of these algorithms is an integral part
of the selection process carried out in~\cite{NIST:LWC}.  In this
paper, we present FELICS-AE, an adaptation of the FELICS
framework~\cite{FELICS:paper} dedicated to authenticated encryption
schemes, which we use to assess the performance of our candidate
\textsc{Lilliput-AE}~\cite{NIST:Lilliput-AE}.

First, in section~\ref{sec:felics}, we will present the original
FELICS framework.  We will then present FELICS-AE in
section~\ref{sec:felics-ae}, going over our work to adapt the
framework and explaining how to set it up, measure algorithm
performance, and add new encryption schemes.  We will present the
results we obtained so far in section~\ref{sec:results}.  To conclude,
we will mention possible improvements for FELICS-AE in
section~\ref{sec:future}.

\section{Background: the FELICS framework}
\label{sec:felics}

The FELICS framework\cite{FELICS:paper} includes a collection of
implementations of encryption algorithms in C, as well as a set of
shell scripts which measure the performance of these algorithms on
various microcontrollers representative of ``Internet of Things''
(IoT) situations.

\subsection{Supported devices}
\label{sec:felics/devices}

FELICS supports the following microcontrollers (\textbf{bolded} words
denote the codenames used within the framework):

\begin{itemize}
\item 8-bit \textbf{AVR} ATmega128,
\item 16-bit \textbf{MSP}430F1611,
\item 32-bit \textbf{ARM} Cortex-M3.
\end{itemize}

The AVR and MSP430 platforms are entirely simulated, which allows one
to measure algorithm performance on these microcontrollers without
physically owning them.  To measure performance on ARM, however,
FELICS requires an Arduino Due board, as well as a J-Link probe.

FELICS can also benchmark algorithms for x86 architectures (codenamed
\textbf{PC}).

\subsection{Metrics}
\label{sec:felics/metrics}

\paragraph{Code size:} FELICS adds up the \texttt{text} and
\texttt{data} sections of an implementation's compiled object code, as
reported by the GNU \texttt{size} program, to measure how much space
the algorithm's footprint on non-volatile memory.

\paragraph{RAM:} to measure the working memory needed by an algorithm,
the framework runs the implementation through a debugger, spraying a
known pattern on the stack before execution and counting how many
bytes were modified after execution.  This figure is added to the
object code's \texttt{data} section.

\paragraph{Execution time:} for simulated devices, FELICS relies on
the simulator to keep track of the number of clock cycles spent on
encryption.  For other devices, FELICS uses specialized assembly
instructions to get this information.

\subsection{Distribution}
\label{sec:felics/dist}

The CryptoLUX wiki\cite{FELICS:wiki} hosts an archive containing the
FELICS's source code (algorithm implementations and benchmarking
scripts).  The wiki also provides detailed instructions to install the
dependencies FELICS needs to compile implementations and measure their
performance on every platform.

The wiki also hosts a virtual machine (32-bit Ubuntu 14.04) where all
dependencies are pre-installed.  This makes it easier to use the
framework since one then does not need to track down all of its
dependencies.

\subsection{Algorithm instrumentation}
\label{sec:felics/adding-algos}

Algorithm implementations must comply with a number of requirements in
order to work with FELICS.  This section presents some of these
constraints.

\paragraph{Encryption and decryption code must be split across
  distinct files.}  When measuring an algorithm's code size, FELICS
outputs three distinct tallies: encryption code size, decryption code
size, and total code size.  To achieve this, FELICS requires
integrators to fill in metadata files, spelling out which object files
are used for encryption, and which are used for decryption.

This means that if an implementation originally had one file featuring
both encryption and decryption functions, an integrator must split it
into two files and tell FELICS which file serves which purpose.  If
the original file contained code used by both encryption and
decryption functions, the integrator must further create a third file
to move the common code to.

\paragraph{Array declarations must be annotated.}  FELICS defines a
set of macros that annotate integer types for two purposes:

\begin{itemize}
\item They specify optimal memory alignment for ingeter arrays arrays:
  this ensures that implementations aliasing byte arrays as
  e.g. 32-bit integer arrays do not accidentally access an array
  member at an address which is not aligned for a 32-bit variable,
  which can degrade performance or cause undefined behaviour.

\item They add the platform-dependent keywords necessary to tell the
  compiler whether arrays should be stored in ROM or RAM: e.g. for AVR
  \texttt{ROM\_DATA\_BYTE} expands to \texttt{const uint8\_t PROGMEM
    aligned}, where \texttt{PROGMEM} instructs \texttt{gcc} to move
  the variable to Program Memory; \texttt{READ\_ROM\_DATA\_BYTE}
  expands to \texttt{pgm\_read\_byte}, which performs the operations
  needed to read from this specific memory region.
\end{itemize}

An integrator must therefore go over an implementation's array
declarations and change their type to the correct macro.

\section{FELICS for Authenticated Encryption}
\label{sec:felics-ae}

FELICS was initially developed to measure the performance of block
ciphers and stream ciphers.  In this section, we describe the changes
we made to adapt the framework to AEAD algorithms; we then describe
how it can be used and extended.

\subsection{Changes from FELICS}
\label{sec:felics-ae/diff-felics}

\subsection{Usage}
\label{sec:felics-ae/usage}

\section{Results}
\label{sec:results}

\section{Future work}
\label{sec:future}

\bibliographystyle{plain}
\bibliography{references}

\end{document}
