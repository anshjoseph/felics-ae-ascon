\documentclass{article}

\usepackage{authblk}
\usepackage{hyperref}


\title{
  FELICS-AE: a framework to benchmark lightweight authenticated block ciphers
}

\author[*]{Kévin Le Gouguec}

\affil[*]{
  Airbus CyberSecurity -
  ZA Clef Saint-Pierre,
  1 Bd Jean Moulin,
  CS 40001,
  MetaPole,
  78996 ÉLANCOURT Cedex -
  France -
  \href{mailto:kevin.legouguec@airbus.com}{kevin.legouguec@airbus.com}
}


\begin{document}

\maketitle

\section{Introduction}
\label{sec:intro}

The CAESAR competition~\cite{CAESAR:submissions} and the NIST
Lightweight Cryptography Standardization Process~\cite{NIST:LWC} have
brought to light several new authenticated encryption schemes
dedicated to ``lightweight'' use-cases.  In these use-cases, target
devices are strongly constrained in terms of computing resources: they
have limited volatile memory (RAM) and non-volatile memory (ROM),
their processors operate at low frequencies and feature few registers,
they may only be able to draw power from a battery that can neither be
recharged nor replaced, etc.

\section{Background: the FELICS framework}
\label{sec:background}

See~\cite{FELICS:paper}.

\section{FELICS for Authenticated Encryption}
\label{sec:felics-ae}

\subsection{Changes from FELICS}
\label{sec:felics-ae/diff-felics}

\subsection{Usage}
\label{sec:felics-ae/usage}

\section{Results}
\label{sec:results}

\section{Future work}
\label{sec:future}

\bibliographystyle{plain}
\bibliography{felics-ae}

\end{document}
